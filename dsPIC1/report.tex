\documentclass[10pt]{article}
\usepackage{fullpage}
\usepackage{fontspec}
\usepackage{xunicode}
\usepackage{xltxtra}
\usepackage{hyperref}
\usepackage{parskip}
\usepackage{caption}
\usepackage[english]{babel}

\title{ELEC-H-473 : dsPIC33 Lab 1 : QA}
\author{CHASTE Gauvain  \\
	Computer Science Student \\
	\and 
	OOMS Aurélien \\
	Computer Science Student \\
}

\date{\today}


\begin{document}

\maketitle

\paragraph{Question 1}
\url{http://ww1.microchip.com/downloads/en/DeviceDoc/70204C.pdf} p. 2

The dsPIC33F/PIC24H CPU has a 16-bit (data) modified Harvard architecture with an 
enhanced instruction set, including significant support for digital signal processing. The CPU 
has a 24-bit instruction word, with a variable length opcode field. The Program Counter (PC) is 
24 bits wide and addresses up to 4M x 24 bits of user program memory space.

A single-cycle instruction prefetch mechanism helps maintain throughput and provides 
predictable execution. All instructions execute in a single cycle, except the instructions that 
change the program flow, double-word move (MOV.D) instruction, table instructions and also 
instructions accessing Program Space Visibility (PSV) take more than one cycle. Overhead-free 
program loop constructs are supported using the DO and REPEAT instructions, both of which are 
interruptible at any point.


\paragraph{Question 2}

Microcontrollers are designed for embedded applications, in contrast to the microprocessors used in personal computers or other general purpose applications.

A microprocessor is an integrated chip with only a cpu, it doesn't have RAM, ROM or other peripheral. they are disigned for multipurpose use such as software develloping, games, document creation, etc.

A microcontroller is designed to perform specific tasks. it's used for working on specific input where a relation with the output is known. The microcontroler has ROM, RAM, and other peripheral. It is used for specific applications like managing keyboard or mouse input.


\paragraph{Question 3}

\url{http://ww1.microchip.com/downloads/en/DeviceDoc/70157C.pdf} p. 36


\begin{tabular}{|l|l|}
\hline
{\bfseries Addressing Mode} & {\bfseries Address Range} \\
\hline\hline
File Register & 0x0000-0x1FFF (see Note)\\
\hline
Register Direct & 0x0000-0x001F (working register array W0:W15)\\
\hline
Register Indirect & 0x0000-0xFFFF \\
\hline
Immediate N/A & (constant value) \\
\hline
\end{tabular}

{\bfseries Note}: The address range for the File Register MOV is 0x0000-0xFFFE
	
\paragraph{Question 4}

\url{http://ww1.microchip.com/downloads/en/DeviceDoc/70157C.pdf} pp. 37-44

File register addressing is used by instructions which use a predetermined data address as an
operand for the instruction. (e.g. the DEC instruction)

Register direct addressing is used to access the contents of the 16 working registers (W0:W15). (e.g the SL instruction)

Register indirect addressing is used to access any location in data memory by treating the
contents of a working register as an Effective Address (EA) to data memory. (e.g. the ADD instruction)

In immediate addressing, the instruction encoding contains a predefined constant operand,
which is used by the instruction. (e.g. the XOR instruction)

An example of instruction using multiple addressing modes is the ADD instruction.


\paragraph{Question 5}

\url{http://ww1.microchip.com/downloads/en/DeviceDoc/70157C.pdf} p. 26

\begin{itemize}
\item 24 - bits instructions
\item 84 instructions -> 7 bits
\item 16 registers -> 4 bits  * 3 registers for register direct and indirect addressing
\item use of immediate addressing mode -> 4 + 5 bits
\end{itemize}

\emph{seems usable / coherent}


\paragraph{Question 6}

Since addresses and immediate values are both 16 bits wide they eat the 24 bits of an instruction quickly.
This cause instruction length problems in case of file register and immediate addressing.


\paragraph{Question 7}

This header file contains every parameter needed to configure the chip. It defines every flag, ports and latches of the environment the microcontroller has access to.


\paragraph{Question 8}

The volatile keyword is used at a variable declaration. It notifies the compiler that the variable is likely to see its value changed without any action taken by the code.


\paragraph{Question 9}\vspace{0.2cm}
\begin{center}
\captionof{table}{INTEGER DATA TYPES}
\begin{tabular}{|l|l|l|l|}
\hline
{\bfseries Type} & {\bfseries Bits} & {\bfseries Min} & {\bfseries Max} \\
\hline\hline
char, signed char 			&		8 	&	-128 		&	127\\\hline
unsigned char 				&		8 	&	0 			&	255\\\hline
short, signed short 		&		16 	&	-32768 		&	32767\\\hline
unsigned short 				&		16 	&	0			& 	65535\\\hline
int, signed int 			&		16 	&	-32768 		&	32767\\\hline
unsigned int 				&		16 	&	0 			&	65535\\\hline
long, signed long 			&		32 	&	$-2^{31}$	&	$2^{31} - 1$\\\hline
unsigned long 				&		32 	&	0 			&	$2^{32} - 1$\\\hline
long long**, signed long long** &	64 	&	$-2^{63}$ 	&	$2^{63} - 1$\\\hline
unsigned long long** 			&	64 	&	0 			&	$2^{64} - 1$\\\hline
\end{tabular}

\captionof{table}{FLOATING POINT DATA TYPES}
\begin{tabular}{|l|l|l|l|l|l|}
\hline
{\bfseries Type} 		&	{\bfseries Bits} &	{\bfseries E Min} 	&	{\bfseries E Max} 	&	{\bfseries N Min} 	&	{\bfseries N Max}\\\hline\hline
float 		&	32 	&	-126 	&	127 	&	$2^{-126}$	&$2^{128}$\\\hline
double* 	&	32 	&	-126 	&	127 	&	$2^{-126}$ 	&$2^{128}$\\\hline
long double &	64 	&	-1022 	&	1023 	&	$2^{-1022}$	&$2^{1024}$\\\hline
\end{tabular}
\end{center}

{\bfseries POINTERS}
All MPLAB C30 pointers are 16-bits wide. This is sufficient for full data space access 
(64 KB) and the small code model (32 Kwords of code.) In the large code model 
(>32 Kwords of code), pointers may resolve to \emph{handles}; that is, the pointer is the 
address of a GOTO instruction which is located in the first 32 Kwords of program space.


\paragraph{Question 10}

\emph{signed}.


\paragraph{Question 11}

It allows better flexibility when porting the code to another processor as well as better readability (sizes are explicit).


\paragraph{Question 12}

When Shakespeare asked, To be, or not to be?, he did not provide the answer. But programming can. Well the answer is FF.

2B |\textasciitilde 2B = FF


\paragraph{Question 13}

The value of a variable is first copied in register 0 (represented by adress 0x0000) then it's copied to the stack in the space allocated to the function where the variable is declared. [0x001c + offset] 0x001c is register w14 (which stores the first address in the scope of the function).


\paragraph{Question 14}

Variables of size bigger than 16 bits are split into different register first then copied to memory from the registers using address stored in w14 and different offsets.


\paragraph{Question 15}

Two threads walk into a bar. The barkeeper looks up and yells, "hey, I want don't any conditions race like time last!"


\paragraph{Question 16}

There are glitches. Since it takes more than one instruction to initialize a variable, variables go through (unwanted) temporary states during their initialization.


\paragraph{Question 17}

No.


\paragraph{Question 18}

\begin{itemize}
\item unused variables : justified but this is not the purpose of this question
\item control reaches end of non-void function : justified but this is not the purpose of this question
\item integer constant is to large : problem, the variables should be large enough to store the values we have in the code
\end{itemize}


\paragraph{Question 19}

It's defined during compile time as a register variable.
It's initialized during execution time and only one instruction is required since the address of a register does not have to be loaded.


\paragraph{Question 20}

\verb!register INT8U *glob1 asm ("w8");!

\emph{"w8" is the destination register}


\paragraph{Question 21}

literal values cannot be stored in the data types used in this program

\begin{itemize}
\item{FP32} 1.27E+50 becomes Infinity (overflow)
\item{FP64} 8.76543E-70 becomes 0 (underflow)
\item{INT16S} -35000 becomes 30536 (wrapping)
\item{INT16U} 0x43210 becomes 12816 (wrapping)
\item{INT8U} wraps
\item{INT8S} wraps
\item{INT32U} wraps
\end{itemize}


\paragraph{Question 22}

In \verb!variable.c! we were only using literal values. Here we have to load the data from the source,
the store it in the destination.


\paragraph{Question 23}

(here wrapping and overflow happens at execution time)

when doing an assignation to a wider variable :

\begin{itemize}
	\item C type coercion automatically extends shorter type to wider types
\end{itemize}

when doing an assignation to a shorter variable :

\begin{itemize}
	\item int variables wrap around : e.g. int8 = int16 \% ($2^8$)
	\item any casting is done after the complete evaluation of an expression
\end{itemize}

when doing an assignation from integer to FP :

\begin{itemize}
	\item the closest matching representation for the integer in the IEEE format is chosen
\end{itemize}


\end{document}